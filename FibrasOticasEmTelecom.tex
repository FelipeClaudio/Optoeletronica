\documentclass[article]{IEEEtran}
\usepackage[brazilian]{babel}
\usepackage[utf8]{inputenc}
\usepackage{cite}
\usepackage{geometry}
\usepackage{graphicx}
\usepackage{amsmath}
\graphicspath{{./images/}}
\begin{document}

%
% paper title
% Titles are generally capitalized except for words such as a, an, and, as,
% at, but, by, for, in, nor, of, on, or, the, to and up, which are usually
% not capitalized unless they are the first or last word of the title.
% Linebreaks \\ can be used within to get better formatting as desired.
% Do not put math or special symbols in the title.
\title{Utilização de fibras óticas em sistemas de telecomunicação}



% author names and affiliations
% transmag papers use the long conference author name format.

\author{
	Felipe C. S. Santos,
	\and
	Thiago K. Lago
	
\IEEEauthorblockA{Universidade Federal do Rio de janeiro \\ 
	Escola Polit\'{e}cnica \\
	Departamento de Engenharia Eletrônica}
}



\IEEEtitleabstractindextext{
\begin{abstract}

\end{abstract}
\begin{IEEEkeywords}
Telecomunicações, fibras óticas, optoeletrônica
\end{IEEEkeywords}}



% make the title area
\maketitle

\IEEEdisplaynontitleabstractindextext
As fibras óticas tem diversas finalidades, sendo uma das mais importantes a utilização em telecomunicações. O avanço das tecnologias de fabricação, modulação e também instrumentação tem tornado cada vez mais viável a utilização das mesmas para transmissões de dados a grandes distâncias com altas taxas de bits. Busca-se através deste paper mostrar o processo de escolha de dimensionamento de uma rede baseada em componentes óticos.
\IEEEpeerreviewmaketitle



\section{Introducão}

\section{Construção da fibra}
Comentar sobre os materiais que são construídos, as janelas de transmissão, os tipos de dispersão, custo-benefício de cada uma delas.

\section{Componentes óticos}
Comentar sobre alguns componentes óticos utilizados como fbg para filtragem dos sinais e amplificadores óticos

\section{Instrumentros de Medida}
É necessário se preocupar também com a qualidade do sinal recebido e a integridade da fibra ótica. Para isto são utilizados alguns equipamentos que permitem fazer a inspeção das mesmas e analisar o sinal recebido.

Ao instalar uma fibra de grande comprimento, a mesma pode sofrer avarias durante o percurso, prejudicando a recepção do sinal. Outro fator que pode ser determinante na qualidade do sinal recebido é a presença de emendas entre os pedaços das fibras. Existem alguns instrumentos utilizados para resolver este problema. Um deles é o OTDR (\textit{Optical Time Domain Reflectometer}).

\subsection{OTDR}
O \textbf{\textit{OTDR}}  utiliza o efeito de retroespalhamento (\textbf{\textit{backscattering}}) dos raios de luz durante a passagem dos sinais luminosos pela fibra ótica. Assim sendo, torna-se possível medir a atenuação do sinal conforme a distância, assim como visto \cite{FOA}

Esse instrumento possui um laser que emite luz em uma frequência pré-determinada e através da diferença de tempo e da potência do sinal medido após o retroespalhamento é possível determinar a relação entre o sinal recebido e a reflexão em uma dada distância de fibra. 

\textbf{Comentar sobre os instrumentos utilizados como OTDR e espectrômetro}

\section{Conclusão}



\end{document}


