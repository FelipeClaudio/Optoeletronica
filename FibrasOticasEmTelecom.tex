\documentclass[article]{IEEEtran}
\usepackage[brazilian]{babel}
\usepackage[utf8]{inputenc}
\usepackage{cite}
\usepackage{geometry}
\usepackage{graphicx}
\usepackage{amsmath}
\graphicspath{{./images/}}
\begin{document}

%
% paper title
% Titles are generally capitalized except for words such as a, an, and, as,
% at, but, by, for, in, nor, of, on, or, the, to and up, which are usually
% not capitalized unless they are the first or last word of the title.
% Linebreaks \\ can be used within to get better formatting as desired.
% Do not put math or special symbols in the title.
\title{Utilização de fibras óticas em sistemas de telecomunicação}



% author names and affiliations
% transmag papers use the long conference author name format.

\author{
	Felipe C. S. Santos,
	\and
	Thiago K. Lago
	
\IEEEauthorblockA{Universidade Federal do Rio de janeiro \\ 
	Escola Polit\'{e}cnica \\
	Departamento de Engenharia Eletrônica}
}



\IEEEtitleabstractindextext{
\begin{abstract}

\end{abstract}
\begin{IEEEkeywords}
Telecomunicações, fibras óticas, optoeletrônica
\end{IEEEkeywords}}



% make the title area
\maketitle

\IEEEdisplaynontitleabstractindextext
As fibras óticas tem diversas finalidades, sendo uma das mais importantes a utilização em telecomunicações. O avanço das tecnologias de fabricação, modulação e também instrumentação tem tornado cada vez mais viável a utilização das mesmas para transmissões de dados a grandes distâncias com altas taxas de bits. Busca-se através deste paper mostrar o processo de escolha de dimensionamento de uma rede baseada em componentes óticos.
\IEEEpeerreviewmaketitle



\section{Introducão}

\section{Construção da fibra}
Comentar sobre os materiais que são construídos, as janelas de transmissão, os tipos de dispersão, custo-benefício de cada uma delas.

\section{Componentes óticos}
Comentar sobre alguns componentes óticos utilizados como fbg para filtragem dos sinais e amplificadores óticos

\section{Instrumentação}
Comentar sobre os instrumentos utilizados como OTDR e espectrômetro

\section{Conclusão}



\end{document}


